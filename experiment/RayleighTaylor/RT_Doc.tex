
\subsection{Rayleigh-Taylor Test}

The Rayleigh-Taylor test simulates a compressible form of the RT instability: lighter fluid which is
supporting (or being accelerated into) denser fluid is locally unstable.

In the inviscid case, any condition of pressure and density gradients pointing opposite ways will be
unstable in this manner.

\subsubsection{Initial Conditions}

Imogen sets up its RT simulation in a rectangular box, with circular conditions on X and Z and a fixed
potential field ('gravity') pulling in the -Y direction. The light/dense contact discontinuity is placed
at $Y = .5$. The parameters are
\begin{itemize}
\item \begin{tt}rhoTop\end{tt} - density of top fluid
\item \begin{tt}rhoBottom\end{tt} - density of lower fluid
\item \begin{tt}P0\end{tt} - the gas pressure at the BOTTOM of the column*
\item \begin{tt}pertAmplitude\end{tt} - the magitude of velocity perturbations
\item \begin{tt}Kx, Ky, Kz\end{tt} - wave vector for coherent perturbations
\item \begin{tt}randomPert\end{tt} - if true, uses random-per-cell velocities instead
\end{itemize}

\subsection{Analysis}

Lord Rayleigh's famous 1883 paper diagnosed the linear instability of the above described situation in the
incompressible case.

If the fluid is inviscid and the change of density is instant (contact discontinuity), the growth rate of
the linear instability in fact increases unboundedly with wavenumber and the system has \textit{zero} lifetime.

Viscosity, finite density gradient, and surface tension all establish short-wavelength cutoffs and regularize
the problem.

