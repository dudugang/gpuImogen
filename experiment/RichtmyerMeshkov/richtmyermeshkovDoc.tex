

\subsection{Richtmyer-Meshkov Instability Test}

RMI occurs whenever a plane shockwave becomes incident upon a non-uniform density interface. 
The plane shock refracts through the interface, imparting differential vorticity that drives
a jet from the denser fluid into the lighter fluid.

\subsubsection{Analysis}

For this run, we used a square grid 2048 zones across, for 20,000 iterations.
\begin{figure*}
\begin{center}
\includegraphics[width=.5\textwidth]{RMI.eps}
\caption{Richtmyer-Meshkov instability at t = 6504}
\end{center}
\end{figure*}

\subsubsection{Initial Conditions}
Boundary conditions are periodic around the X axis, constant around the Y axis, and mirrored around the Z axis.

The physical input parameters to the Richtmyer-Meshkov Instability test are:
\begin{itemize}
\item \tt{mach} - Defines the mach speed of the incoming shock wave
\item \tt{rhotop} - Defines the density of the lighter fluid 
\item \tt{rhobottom} - Defines the density of the heavier fluid
\end{itemize}

To create the non-uniform interface, we create a cosine wave with an amplitude 1/20 of the height 
of the grid and with a wavelength 2x the length of the grid. We then place the center of the wave 
4 total wave amplitudes below the center of the grid, and set the mass density of the entire 
region below the wave equal to rhobottom. The rest of the grid is previously set to have density 
equal to rhotop, and is unchanged by the wave. We then create a shocked region from the top of 
the grid to a height just 20 zones above the peak of the wave. This shock is composed of a momentum 
and pressure equal to a blast wave with mach set by the input 'mach'.

Note that the entire region must then be given a momentum in the direction opposing the shock to 
maintain position on the grid. This opposing momentum is calculated by the same function as is used 
to calculate the initial shock.
