

\subsection{Centrifuge Test}

The centrifuge test checks the ability of the code to evolve an analytically time-independent
hydrodynamic flow and maintain its original time-independent values, and also provides a
valuable test of the rotating frame's implementation.

\subsubsection{Initial Conditions}

The physical input parameters to the centifuge test are:
\begin{itemize}
\item \begin{tt}rho\_0\end{tt} specifies the density at either the center (Isochoric) or edge (isothermal
or adiabatic) of the simulation region
\item \begin{tt}P\_0\end{tt} specifies the pressure at the same points. The isothermal $c_s$ and adiabatic $k$
are defined through the pressure.
\item \begin{tt}omegaCurve\end{tt} is an anonymous function of an array $r$ which gives $\omega(r)$. This is 
assumed to obey the constraints given. $r$ is normalized between 0 (center) and 1 (edge).
\item \begin{tt}eqnOfState\end{tt} must be symbolically set to one of the \tt{CentrifugeInitializer} class' 
symbolic values \tt{EOS\_ISOTHERMAL}, \tt{EOS\_ISOCHORIC} or \tt{EOS\_ADIABATIC}.
\end{itemize}

Two additional numeric parameters are
\begin{itemize}
\item \begin{tt}edgeFraction\end{tt} - Sets how far the simulation domain extends past the edge of the
rotating region, as a multiple of the radius of the centrifuge: $2(1+e) = Nx$
\item \begin{tt}frameRotateOmega\end{tt} - Sets the rate at which the simulation frame is rotated. This can improve the 
timestep considerably if rotation is in one direction and advection speed is high. In fluid velocities,
positive $\omega$ about $\hat{z}$ is such that particles on $+\hat{x}$ have momentum $+\hat{y}$. The fluid frame is opposite, such
that positive \begin{tt}frameRotateOmega\end{tt} will create the appearance of points on $\hat{x}$ swinging towards $-\hat{y}$.
\end{itemize}

\subsubsection{Analysis}

The centrifuge test balances centripetal acceleration against a pressure gradient,
\[ \rho r \omega(r)^2 = dP/dr \]
Under the ideal gas law,
\[ P = K \rho T \]
and assuming that we are dealing with a thermodynamically simple gas or mixture whose 
internal degrees of freedom are pleasantly unchanging,
\[ \rho r \omega(r)^2 = K T d\rho/dr + K \rho dT/dr \]

Two of $\omega$, $\rho$ and $T$ have to be specified, then our differential equation
plus boundary conditions defines the third.

We chose to define an arbitrary $\omega$, except for the requirement that velocity behave
itself at the origin ($\omega$ diverging slower than $1/r$) and with the awareness that
 $\omega$ will only be evaluated on a compact region normalized to $[0,1]$, outside of
which it is set to zero (in other words, fluid at rest at infinity).

A pressure-supported structure can't be isobaric but isothermal, isochoric and adiabatic equilibria
can all be defined for any sane initial conditions. The ODE is solved by separation
of variables; Because it occurs often, the quantity
\[ \int_{r_0}^r r \omega(r)^2 dr \equiv \Phi(r,r0) \]
to save space.

\textbf{Isothermal}
With a fixed temperature (represented in the code as a fixed isothermal soundspeed), the ODE
\[ \Phi(r,r_0) = K T \int_{\rho_0}^\rho d\rho / \rho \]
is separated and has solution
\[ \rho(r)  = \rho_0 e^{\Phi(r,r_0)/KT} \]
With $\rho_0$ specified at the outer edge radius $r_0$ and an isothermal soundspeed $KT$,
a physical solution exists for sane inputs.

\textbf{Isochoric}
At a fixed volume (here taken as fixed density), the ODE
\[ \Phi(r,r_0) = K \int_{T_0}^T dT \]
is separated and has solution
\[ T(r) = (a^2 + \Phi(r,r0))/K \]
which gives pressure
\[ P = \rho K T = \rho_0 (a^2 + \Phi(r,r0)) \]
With the initial temperature set by proxy by isothermal soundspeed $a$ at the center
and density fixed, the temperature ramps up as required and a solution exists for sane
inputs.

\textbf{Adiabatic}
Under adiabatic conditions we use the relation $P = K \rho^\gamma$ and so
\[ \frac{\Phi(r,r_0)}{K \gamma} = \int_{\rho_0}^\rho \rho^{\gamma-2} d\rho \]
with solution
\[ \rho(r) = \left[ \rho_0^{\gamma-1} + \frac{(\gamma-1)\Phi(r,r_0)}{K \gamma} \right]^{1/(\gamma-1)} \]
Defining $\rho_0$ at $r_0$ and given $K$, a solution exists for all sane inputs; Imogen
defines it at the outer edge.


